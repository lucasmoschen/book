\chapter{Introdução}
%A linguagem matemática é caracterizada por seu rigor formal, e possibilidade de abstração sobre diferentes contextos.
%Somos capazes de descrever situações impensáveis, e obter respostas para problemas que seriam impossíveis sem o uso desse rigor.

%Particularmente, não sei quanto dessa exposição teórica é relevante. Certamente é importante num curso de matemática; mas como a matéria tem um enfoque muito na Lógica (que não é Matemática), não sei se isso fica adequado. Acho que vou deixar por é melhor pecar pelo excesso que pela falta.

Há evidências do uso da matemática já por volta de 3000 a.c, onde povos antigos como os mesopotâmicos, egípcios e moradores de Ebla utilizavam a aritimética, álgebra e a geometria para taxação, comércio, trocas, astronomia e na formulação de calendários e marcação do tempo corrente.
Os registros textuais mais antigos desse tipo de prática vem da Mesopotâmia e do Egito, e são datados aproximadamente entre 2000 a.C e 1800 a.C, mas muitos estudiosos consideram a Grécia Antiga como o berço da matemática, onde, por volta de 600 a.c, os Pitagóricos iniciaram o estudo da matemática como uma ``disciplina demonstrativa" e introduziram o raciocínio dedutivo e o rigor matemático em provas, sendo que a forma como se escrevia matemática era bem diferente da usual atualmente, com uso extensivo de símbolos para abstrair grandezas e representar variáveis.

Paralelo ao desenvolvimento da Matemática, Aristóteles, estudando Retórica, percebeu padrões na formulação frasal humana e cognitiva da linguagem. De fato, o clássico exemplo
\vspace{0.3cm}

Todos os homens são mortais. \par
Socrates é homem. \par
Logo, Sócrates é mortal.

\vspace{0.3cm}


Ilustra um mecanismo dedutivo (modus ponens, que é visto mais à frente) que fazemos naturalmente e é estudado a fundo na sua obra, que é o marco zero dá lógica clássica, que está diretamente relacionada à oposição do verdadeiro e do falso. Essa visão imperou até o fim do século XIX, quando percebeu-se limitações abstratas disto e surgiram esforços no sentido da formalização da Matemática com Frege, Gentzen e outros (esse campo ficou denominado metamatemática).

Apenas quando surge o questionamento de \textit{como sabemos}, e não apenas \textit{o que sabemos} as bases para desenvolver um conhecimento sólido são devidamente colocadas na Matemática.
Aqui, a lógica formal entra exatamente ao questionar \textit{como sabemos}, ou melhor, como temos certeza do que concluimos partindo de um contexto inicial.
A formalização do processo de raciocínio garante que \textit{construimos um castelo sólido}.
%A filosofia, o método científico e o formalismo matemático que conhecemos se devem em grande parte a essas bases bem definidas.

Não devemos, no entanto, nos limitar a discutir lógica como ferramenta relacionada a teoremas ou metafísica. Esse ramo é presente no estudo da linguagem (lógica informal), processos industriais e computação pela verificação de hardware e de software (métodos formais), e estudos em representação do conhecimento, por exemplo.

\section{Lógica formal e linguagem natural}
% http://www.pucrs.br/edipucrs/online/pesquisa/pesquisa/artigo11.html
A linguagem humana é conhecidamente um emaranhado gigantesco de simbolos, gestos, palavras, sentidos, interpretações.
Essa é fruto de lapidação através de milênios, do que se iniciou com uma linguagem básica animalesca, chegando a complexidade atual em um processo evolutivo constante; assim ocorreu com a fala e escrita.

É impossível afirmar, portanto, que uma língua, digamos o português, é ineficiente na transmissão de significados, sendo fruto de evolução pura e constante.
De fato, os linguistas se dedicam em grande parte a desvendar os mecanismos segundo os quais a lingua se molda e se torna tão eficiente, transmitindo significados profundos, muitos dos quais difíceis de serem explicados formalmente. A linguagem natural permite representar diversas noções como medo, incertezas e possibilidades; definições estas extremamente imprecisas.

Em contraparte, linguagens formais buscam firmar o processo de pensamento sob regras bem definidas. Muitas vezes, essas linguagens não captam as nuances da linguagem natural em sua completude, no entanto permitem abstrair conceitos importantes, eventualmente complexos para analisar numa linguagem naturalmente imprecisa.
Por exemplo, podemos afirmar que \textit{toda vaca voadora come gente}, e isso não fará o menor sentido a não ser que estabeleçamos essa como uma afirmativa lógica ao abstrair a realidade (tente explicar a alguem por quê essa frase está correta!).
Esse é um exemplo caricato que esclarece como a lógica nos permite discutir conclusões acertadas, que seriam pouco claras apenas utilizando de um arcabouço de linguagem natural ou da cognição humana.

Mais uma vez, é importante reforçar que a lingua é um meio extremamente eficiente de transmissão de significados.
Há um ramo da lógica chamado lógica informal que lida com o modo como expressamos o raciocínio através da lingua: argumentação e falácias, por exemplo. \textit{Um ser humano sem linguagem seria capaz de entender a realidade? Enlouqueceria?} Aqui entramos em uma discussão distante ao escopo do livro, sobre a própria filosofia da linguagem, antropologia e mesmo biologia.

\section{Sobre Sistemas Dedutivos}
% Um pouco sobre sistemas, DN, Cálculo de Sequentes e Resolução, pex.
Para motivação, considere o seguinte problema: \textit{Você acorda em uma sala com duas portas. Uma dá para a liberdade, e a outra leva a morte, mas você não sabe qual é cada uma.
Junto a cada porta há um guarda. Um dos guardas é sempre sincero, enquanto o outro é sempre mentirozo. Você tem direito a uma única pergunta para decidir qual porta tomar. O que você faz?}

Esse é claramente um problema envolvendo um raciocínio complexo, e, se tratando da sua vida em jogo (vamos fingir isso), exige plena certeza da solução antes de uma resposta definitiva.
De fato, seremos capazes de formalizar problemas desse tipo, e obter métodos de derivação para obter um prova para a resposta. Essa formalização se dará em dois níveis: o semântico e o sintático. O primeiro tem relação direta com nossa interpretação sobre os objetos que manipularemos, definidos como \textit{"fórmulas bem formadas"} mais à frente. O segundo está relacionado com regras de derivação e inferência que operamos sobre esses objetos e (normalmente desejamos) que preserve relações da sua derivabilidade com a corretude a ser definida semânticamente. Esse conjunto de regras de inferência e axiomas que operam diretamente sobre as fórmulas para provar teoremas é chamado de \textbf{sistema dedutivo}.

Falaremos de três sistemas, como exemplos: a dedução natural, o cálculo de sequentes e o de resolução.

A \textit{dedução natural} aparecerá frequentemente ao longo do livro. É uma evolução de um sistema mais antigo (o sistema de Hilbert, que usava muitos axiomas e poucas regras de inferência; com a dedução natural vai ser o contrário). As provas terão formato de árvore (o que é de certo modo natural, se pensar na forma como se faz a prova é possível ver que há vários desenvolvimentos paralelos que chegam num ponto, o teorema a ser provado; isso inerentemente tem \textit{cara de árvore}). Como o leitor poderá perceber mais à frente, as provas em dedução natural pode variar muito de tamanho dependendo das nossas premissas e do que nós queremos provar.

O \textit{cálculo de sequentes} é uma espécie de \textit{dedução natural multidimensional}. Foi criado pelo Gentzen também para aperfeiçoar a dedução natural e tratar melhor a diferença da expressão intuicionista da clássica na dedução natural (isso vai ser explicado em outro momento).

Por fim, \textit{resolução} é basicamente um princípio dedutivo que possui uma regra de unificação de duas fórmulas e consegue assim pegar diversos axiomas e o teorema que voçê quer provar, reescrevê-los para que essas regras de unificação possam ser feitas e entrar em contradição com o que se deseja provar (isso fará sentido em algum momento).

%\subsection{Dedução Natural}
% Com que nível de detalhe nós devemos abordar as regras??

%Uma das maneiras de formalizar e demonstrar problemas lógicos é utilizar sistemas dedutivos como a Dedução Natural, que consiste em um grupo de regras e axiomas que nos permitem manipular expressões formalizadas, ou seja, traduzidas da forma linguística para uma forma lógica matemática de notação, estabelecendo a validade dos argumentos, derivando a conclusão do argumento a partir das premissas usando esse sistema de regras em questão.
%Por meio dessas regras de inferência, podemos demonstrar a validade de uma infinidade de fórmulas e argumentos sem a necesidade de considerar os valores que cada fórmula ou subfórmula recebe, ou seja, não estamos mais lidando com a semântica, mas com a sintaxe.

% Ver o que precisamos deixar aqui e o que colocar em cima


%Uma outra característica da Dedução Natural é que ela possui um certo formato para suas demonstrações, onde as provas são apresentadas de forma que cada linha corresponda a um passo da prova ou demonstração, as colunas correspondam as premissas e abaixo dessa linha teremos a nossa conclusão.
%Como o leitor poderá perceber mais à frente, as provas em Dedução Natural pode variar muito de tamanho dependendo das nossas premissas e do que nós queremos provar.

%\subsection{Cálculo de Sequentes}

%O Cálculo de Sequentes é um estilo de apresentação de provas que foi proposto por Gentzen, em 1934, com o objetivo de servir como uma ferramenta auxiliar para o estudo de dedução natural em lógica de primeira ordem.
%Assim como outros sistemas dedutivos, o Cálculo de Sequentes também possui suas regras e propriedades que funcionam de maneira harmônica e mantém o aspecto de árvore para as provas em si.

%O estudo do cálculo de sequentes é especialmente importante para a melhor compreensão de algumas lógicas não-clássicas em especial as lógicas sub-estruturais.
%Assim como outros sistemas dedutivos,
%o Cálculo de Sequentes também possui suas regras de manipulação e propriedades que funcionam de maneira harmônica e mantém o aspecto de árvore para as provas em si.
%Essas regras são divididas em dois tipos: lógicas e estruturais.
%As regras lógicas nos mostram como introduzir conectivos lógicos a esquerda e a direita em um sequente enquanto as regras estruturais,
%como o próprio nome diz são estruturais.


%Não acho que falar LK e LJ seja útil. LK se refere ao cálculo de sequentes para a lógica clássica e o LJ para a intuicionista.
%Visto que isso foi pouco abordado na sala, entrar nesse nível de detalhe pode ser meio inadequado.

%Tirei LK E LJ
%Colocaremos as regras de cálculo sequentes?

%\subsection{Resolução}
%CNF, regra da resolução
%talvez explicar superficialmente e dar um exemplo no capitulo de lp (não vamos falar de resolução em primeira ordem; embora exista a gente não estudou né)

% O método da resolução foi introduzido por John Alan Robinson que foi um filósofo, matemático e cientista da computação que foi um dos pioneiros em lógica de programação e fez importantes descobertas e grandes contribuições para a área de fundamentos
%de provadores de teoremas automáticos, os ATPs (Automated theorem proving).
%O uso destes provadores automáticos será abordado na próxima seção.

% O sistema dedutivo de resolução faz uso de regras de inferência afim de demonstrar por refutação sentenças e inferências da lógica proposicional e da lógica de primeira ordem.
%Este sistema dedutivo usa a linguagem proposicional no formato CNF (Conjunctive Normal Form) e nâo possui axiomas, mas apenas uma regra de inferência que opera com cláusulas e origina uma nova implicada por elas.


\section{Provadores, ATP e ITP}
% referencia que bem diferencia ATPs e ITPs
% além de direcionar o que desejamos sobre provadores
% https://leanprover.github.io/theorem_proving_in_lean/introduction.html

Verificação formal envolve o uso de lógica e, mais recentemente, de aparato computacional na descrição de um contexto em termos matemáticos suficientemente precisos, e estabelecimento de assertivas sobre essas definições.

A partir disso, passamos utilizar teoremas para verificar assertivas sobre os elementos abstraídos, que podem bem representar todo tipo de objeto de interesse: somos capazes de descrever \textit{softwares}, processos industriais, sistemas de gerenciamento de cargas, ou teoremas sobre conjuntos, por exemplo, utilizando dos mesmos aparatos.
Isso significa que na prática não há distinção entre ferramentas provando teoremas, ou verificando assertivas sobre um objeto qualquer.
Reduzimos os problemas de verificação formal a problemas de provas de teoremas e, mais relevante, podemos utilizar computadores nessa tarefa.

Esse tipo de discussão se tornou possível após os recentes avanços no campo de estudo da lógica.
%Reduzimos regras de inferencia e derivação presentes na maior parte das provas a um conjunto pequeno de axiomas fundamentais.
Provadores automáticos se tornaram viáveis com o desenvolvimento por Frege de uma base formal adequada a descrição da matemática, em sua \textit{Begriffsschrift}.% fale isso 5 vezes sem engasgar!.
Após Frege, outros nomes, como Alas, Russeau, Hilbert, e Gentzen se dedicaram a desenvolver bases compactas para o pensamento matemático que viriam a ser implementadas pelos computadores.

A compactação do conjunto de regras de demonstração viabilizaria computadores implementando ferramentas para auxílio a prova de teoremas, divididas em basicamente duas classes: auxílio a contrução de uma prova, ou verificação de uma assertiva dada. Esses são os \textit{ATPs} e os \textit{ITPs}.

\subsection{Provadores de Teorema Automáticos}
% falamos sobre os ATPs, alguns exemplos... Descrevemos características.
% https://en.wikipedia.org/wiki/Automated_theorem_proving
% https://www.cs.cmu.edu/~fp/courses/atp/handouts/atp.pdf

Os provadores automáticos (\textit{Automated theorem proving}) são uma classe de provadores assistentes e visam a obtenção ou verificação de um teorema.
Portanto, pertencem a uma classe voltada a obtenção de um termo que representa uma prova, ou a verificação de um teorema (o que igualmente se reduz a tarefa de obtenção de uma prova).

Dessa forma, um \textit{ATP} pode tomar como entrada um teorema, e deveria ser capaz de dar um retorno do tipo \textit{verdadeiro} ou \textit{falso}, uma prova ou contraexemplo para a assertiva.
Note que nesse processo, a participação do ser humano está restrita apenas a aplicação da entrada, ou a descrição do problema para o computador.
Isso expressa o sentido em que esses provadores são \textit{Automáticos}.

De fato, algumas provadores implementam sistemas dedutivos que lidam internamente com o problema, desenvolvendo soluções pouco inteligíveis (resolução e variações algorítmicas da resolução para melhora de performance são as principais formas de implementar isso).
Pode-se dizer que o processo de prova nesses casos é pouco acessível. Exemplos de ATPs populares são:

\begin{itemize}
	% E prover
    \item \textbf{E theorem prover\footnote{https://wwwlehre.dhbw-stuttgart.de/~sschulz/E/E.html}} : suporta lógica de primeira ordem com igualdade. Aplica transformação de fórmulas num forma normal e vai aplicando a regra da resolução e para se achar uma contradição (eles negam o que se quer provar e colocam como axioma; se chegar em falso é porque o teorema vale, se não, não vale).
    \pagebreak
    \item \textbf{Vampire\footnote{www.vprover.org}}: é um poderoso provador automático implementado em \textit{C++}, para lógica de primeira ordem. Seu desenvolvimento se iniciou em 1994, e já venceu a \textit{copa mundial de provadores de teoremas}, além de diversas premiações menores.
    \item \textbf{SNARK\footnote{http://www.ai.sri.com/~stickel/snark.html}}: um provador automático implementado em Lisp desenvolvido pelo SRI. Já foi usado pela NASA e em alguns outros projetos grandes. A documentação está meio desatualizada, mas sua vantagem prática é que pode ser chamado num ambiente LISP e aceita uma escrita diferente.
    %refiro-me ao foto de não ser obrigado entrarem as fórmula nas linguagens tptp
\end{itemize}

Esses dois provadores implementam mecanismos similares de resolução. Além disso, os provadores automáticos (quase) todos suportam um formato de escrita de fórmulas numa sintaxe baseada em Prolog, que são as famílias de linguagens TPTP\footnote{http://www.tptp.org/}. Uma vantagem do SNARK é que ele permite uma escrita mais simples que essa, além de aceitar as TPTP.

Em 1976, o computador IBM360 participou da prova dada por Apel e Haken para o \textit{Teorema das quatro cores}, mostrando que o uso de computadores poderia ser relevante no desenvolvimento da matemática (a prova desse teorema é baseada em exaustão, mas são milhares de casos; certamente a automação vira uma mão na roda nesses casos).

\subsection{Provadores de Teorema Interativos}
% Damos características desse paradigma, exemplos...
% https://www.cl.cam.ac.uk/~jrh13/papers/joerg.pdf

Os provadores interativos (\textit{Interactive theorem proving}) representam uma classe de ferramentas que possibilitam máquina e humano trabalhando juntos no desenvolvimento de uma prova formal.
Nesse contexto, a ferramenta é comumente usada na verificação do processo de prova desenvolvido por um indivíduo.

A intervenção da máquina pode ser superficial, como, por exemplo, numa checagem dos teoremas utilizados em um processo de prova, ou mais intensa, em que lacunas são preenchidas ou completadas por etapas de automação.

A vantagem nesse tipo de método está presente justamente quando trabalhamos provas que se tornam extremamente complicadas, e um provador automático se torna incapaz de solucionar, ou quando desejamos ter certeza de um resultado, pela dificuldade em garantir a validade do resultado dado por um \textit{ATP} pela dificil interpretabilidade dos seus outputs.
Problemas desse tipo abrem espaço para o paradigma que garante provas desenvolvidas pela intervenção humana, com assistência computacional.

Alguns \textit{ITPs} populares são:
\begin{itemize}
    \item \textbf{Coq Proof Assistant\footnote{Coq: coq.inria.fr}} : sistema de gestão de provas que permite exprimir assertivas matemáticas, checagem de provas, além de executa algumas provas automáticas. Embora permita essa forte automação, é ainda considerado sistema de provas interativo, oferecendo um ambiente para provas checadas por máquina. Tem uma comunidade bem grande e adquiriu popularidade crescente após ser usado pra provar o teorema das quatro cores.

    \item \textbf{Isabelle\footnote{Isabelle: isabelle.in.tum.de}}: desenvolvido em parceria entre a Universidade de Cambridge e a Universidade Tecnica de Munique, Isabelle foi pensada para servir de assistente de provas geral, implementando lógica de alta ordem. Teve sua versão inicial lançada em 1986.
\end{itemize}

Tem outros que valem a pena ser lembrados, como o Agda e o ACL2, sendo este bem estabelecido para própositos de verificação de hardware. Mais a frente discutioms como o Lean implementa um plataforma entre os dois paradigmas: permite o desenvolvimento de provas interativas, sem abrir mão da automação.
Podemos discutir \textit{em que ponto o sistema deixa de agir como assistente, e passa a ser considerado autônomo}, ou ainda, \textit{o que é automação}.

%\subsection{O que é Automacão}
%fiquei interessafo em defender o que discutimos tantas vezes com o Rdmkr, sobre até onde, e como, o computador está automatizando as tarefas...
%é. Acho que isso a gnt só vai descobrindo com mais uso msm (acho que até agr n sabemos bem totalmente sobre isso)
% que tal uma discussão sobre máquinas desenvolvendo conhecimento, e eventualmente explosão da inteligencia artificial de filmes!? Ou pelo menos, sobre o medo do público geral sobre esse tipo de tecnologia...

% \section{Resumo}
% só porque pode ser interessante. Muito agradável ter um resumo no final de uma capítulo introdutório.

%\section{o que esperamos do livro}
% ainda extremamente razoável: quais tópicos, etc...
